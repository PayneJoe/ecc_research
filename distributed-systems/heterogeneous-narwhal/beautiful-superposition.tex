\scriptsize
\begin{tikzpicture}[scale=1.5]
\foreach \colour/\flip/\sh in {red/1/0,blue/-1/6} {
  \begin{scope}[opacity=.4,shift={(\sh,0)},xscale=\flip]
          % create blocks round by round ... 
          \foreach \rr in {1,...,4}{
            \foreach \v in {1,...,\value{theN}}{
              \pgfmathtruncatemacro{\showon}{\v+(\rr-1)*\value{theN}}
              % \uncover<\showon->
              {%
                \node[draw,rectangle,color=\colour,inner sep=1pt] (b\v-\rr) at (\v,\rr) {\phantom{m}\faBox[duotone]\phantom{m}};% %{\(b_{\v,\rr}\)};%
                \ifthenelse{\equal{\rr}{1}}%
                {%
                  \node[color=\colour,anchor=north,inner sep=1ex] (srv_\v-\rr) at ([yshift=-2ex]b\v-\rr.south) {\large\color{black}\faServer};

                  \pgfmathtruncatemacro{\rednumber}{\v}
                  \pgfmathtruncatemacro{\bluenumber}{6-\v}
                  \ifthenelse{\equal{\colour}{red}}
                  {\node[anchor=north,inner sep=0pt,outer sep=0pt] at (srv_\v-\rr.south) {%
                      \makebox[0pt][r]{\color{red}%
                        \Large\(\bullet\)}\ensuremath{\rednumber}%
                    };}
                  {\node[anchor=north,inner sep=0pt,outer sep=0pt] at (srv_\v-\rr.south) {\ensuremath{\bluenumber}%
                      \makebox[0pt][l]{%
                        \Large{\color{blue}\(\bullet\)}%
                      }%
                    };}
                }%
                {}%
              }
            }
          } % end of first \rr 
    %       % ... and the links of certificates of availability "CoA-links" / cf. signed quorums
          \foreach[remember=\rrr as \lastr (initially 1)] \rrr in {2,...,4}{
            \foreach \v in {1,...,\value{theN}}{
              \pgfmathtruncatemacro{\showon}{\v+\lastr*\value{theN}}
              \foreach \l in {1,...,\value{theQ}}{
    %             % \uncover<\showon->
                 {%
                  \node[circle,fill=\colour,inner sep=-1pt] (b\v-\rrr_\l) at ([xshift=\l ex-.5ex]b\v-\rrr.south) {\tiny\,};
                  \draw[thick,color=\colour,double,opacity=.4]
                  (b\v-\rrr_\l.center) 
                  .. controls ++(0,-.5) and ++(0,.5) .. %
                  (b\arabic{linkToBlockFrom\v,\rrr,\l}-\lastr.north) %
                  % node[pos=.5]{
                  % \l
                  % }
                  ;
                 }
               }
             }
           }
           \end{scope}
         }
       \end{tikzpicture}


%%% -output-format=dvi ← for externalizing dvis instead of pdfs
%%% Local Variables:
%%% mode: latex
%%% TeX-master: "HeterogeneousNarwhal"
%%% TeX-engine: luatex
%%% TeX-command-extra-options: "-shell-escape"
%%% End: