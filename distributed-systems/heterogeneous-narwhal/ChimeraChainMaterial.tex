\section{Motivation / Background / Context}

\subsection{Atomic cross-chain swaps}
\label{sec:atom-trans-across}

\begin{quote}
  An atomic cross-chain swap is 
  a distributed coordination task 
  where multiple parties exchange assets across multiple blockchains, 
  for example, trading bitcoin for ether.

  An atomic swap protocol guarantees 
  (1) if all parties conform to the protocol, 
  then all swaps take place, 
  (2) if some coalition deviates from the protocol, 
  then no conforming party ends up worse off, and 
  (3) no coalition has an incentive to deviate from the protocol.
\hfill\cite{podc18Herlihy} 
\end{quote}

\subsection{Let's start in the land of naïveté}
\label{sec:naivities}

\subsubsection{The impossible}
\label{sec:impossible-concurrency}

One simple ``ideal'' solution for cross-chain swaps is 
``concurrent''  inclusion of 
transactions in two blocks, one on each chain, 
which refer to each other reciprocally.
However, 
this simply cannot be accomplished with hash links.

\subsubsection{A seemingly good situation}
\label{sec:almost-safe}

Another naïve approach consists in
% in a \emph{combo block}  
% is proposing the next ``same'' block on two chains,
% at the same “moment”,
% e.g., if on 
is running a validator on both involved chains. 
Whenever the validator has the right to propose 
the next block on both chains,
we “simultaneously” propose one transaction on each chain. 
However,
we could “loose” the proposal spot,
\eg if a view change is initiated on one of the two chains.
\todo{%
  describe in more detail 
}
Even worse,
it could be that one of the proposals goes through,
while the other gets “vetoed”.

\subsubsection{The envisioned solution: chimera chains}
\label{sec:envisioned-solution}
We propose to establish a ledger that records transactions
that involve two (or more) pre-existing chains%
---dubbed \emph{chimera chain}. 
If we want to speak in terms of bridges,
our envisioned solution combines the ideas of trusted and trustless bridges.
 \todo[inline]{%
  add fine print on which bridges might actually do this already:
  in particular, trustless bridges,
  see \eg \url{https://ethereum.org/en/bridges/}
}

The chimera chains orders blocks with cross-chain transactions
such that inclusion in 
the chimera chain leads to (eventual) execution in the two \base[s]. 
The seemingly simple principle for running the chimera chain
is maximal usage of pre-existing validator sets.
\endnote{%
  though additional validators might want to join for 
  the purpose of bridging for the purpose of chimera chains
}
In particular, we want
\begin{enumerate}
\item more than a single validator in the intersection of two quorums
\item a protocol, 
  that gives these validators the opportunity to propose \emph{combo block}
  on a chimera chain. 
\end{enumerate}
\todo{
  explain in non-technical terms that
  the same-slot-on-both-chains situation
  has a very intuitive counterpart in
  having quorum-pairs between 
  two chains as the quorum system for the chimera chain 
  of two chains. 
}\endnote{%
  for “quorum-pairs” between two chains,
  explain why it is still much cheaper
  than having a merged super-chain
  (if it is ...)
}


\subsection{Chimera chains}
\label{sec:chimera-chains}

% \paragraph{Overview}

Assuming a fixed set of \base[s], 
each of which have a set of quorums
such that one of them “should” be live at any point in time,%
\footnote{%
  This is a simplifying assumption,
  which we use only in the exposition.
  In principle,
  learners are completely independent entities.
  They “dictate” which quorums they would trust,  
  in particular concerning liveness.
  For safety, %
  the situation is more intricate. %
}%
\todo{explain this}\xspace%
chimera chains have the following phases:
\begin{description}
\item[Spawn] %
  Among other things,
  the genesis blocks are generated,
  including links to the blocks in all \base[s]
  that “summoned” the chimera chain. 

\item[Active]
  The usual mode of operation,
  in particular facilitating cross-chain transfers. 

\item[Retire] 
  Low activity or inconsistencies might call 
  for retirement of the chimera chain. 

\item[Merge]
  In principle,
  the ecosystems of two \base[s] do “completely” merge
  such that the chimera chain is still performant enough.
  In fact,
  this might be a good thing:
  smaller \base[s] are just joining larger ones.
  Clearly,
  there's a trade off,
  between centralization and usability. 
\end{description}

The chimera chain has dedicated “administrative” transactions,
e.g.,
for spawing










